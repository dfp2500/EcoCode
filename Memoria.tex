\documentclass[12pt,a4paper]{report}

% Paquetes comunes
\usepackage[utf8]{inputenc}
\usepackage[spanish]{babel}
\usepackage{amsmath}
\usepackage{graphicx}
\usepackage{hyperref}
\usepackage{fancyhdr}
\usepackage{geometry}
\usepackage{tocbibind}
\geometry{top=2.5cm,bottom=2.5cm,left=2.5cm,right=2.5cm}

\title{EcoCode: Una herramienta para el calculo de la huella de carbono del software}
\author{Daniel Fernández Jiménez}
\date{\today}

\begin{document}

% Título
\maketitle

% Índice
\tableofcontents
\newpage

\chapter{Introducción}

\section{La importancia de ser conscientes de la huella de carbono del software}
En nuestra sociedad tanto la tecnologia como el software del que dispone, son de uso cotidiano, necesario en muchos aspectos y esta en constante evolución. Sin embargo, el impacto ambiental de esta tecnología 
es un tema que ha cobrado cada vez más relevancia en los últimos años. La creciente preocupación por el cambio climático ha llevado a gran variedad de sectores a buscar formas de reducir su huella de carbono, 
y el software no es una excepción.

El software, aunque intangible, tiene un impacto medioambiental significativo debido a su consumo de energía y los recursos necesarios para su desarrollo, implementación y mantenimiento. 
La huella de carbono del software se refiere a la cantidad total de emisiones de gases de efecto invernadero (GEI) asociadas a su ciclo de vida, desde la producción hasta el uso y la eliminación.
La necesidad de evaluar y reducir la huella de carbono del software es crucial para contribuir a la sostenibilidad ambiental. A medida que la demanda de software aumenta, también lo hace su impacto ambiental. 
Por lo tanto, es fundamental que los desarrolladores y las organizaciones sean conscientes de este impacto y busquen formas de mitigarlo.

Reducir la huella de carbono del software no solo es beneficioso para el medio ambiente, sino que también puede resultar en ahorros económicos y una mejora en la eficiencia operativa ya que un software más eficiente consume menos recursos y energía.
Esto puede traducirse en menores costos operativos y un mejor rendimiento del sistema. Además, las organizaciones que demuestran un compromiso con la sostenibilidad pueden mejorar su reputación y atraer a clientes y empleados que valoran la responsabilidad ambiental.

\section{Objetivo General}
El objetivo principal de este trabajo es desarrollar una herramienta que permita calcular la huella de carbono generada por un sistema software, teniendo en cuenta tanto el gasto energetico realizado por el software, su traduccion a Co2eq dependiendo de la fuente de energia utilizada y su gasto asociado

Además, se ofrecerán recomendaciones prácticas sobre cómo reducir la huella de carbono generada por el software, en función de los resultados obtenidos. Estas recomendaciones estarán orientadas tanto a desarrolladores individuales como a equipos de desarrollo, y se enfocarán en mejorar la eficiencia energética, optimizar el uso del hardware y adaptar las aplicaciones a entornos más sostenibles.

\section{Objetivos Específicos}
Para lograr el objetivo general, este trabajo se desglosa en los siguientes objetivos específicos:

\begin{itemize}
    \item \textbf{1:} 
    \item \textbf{2:} 
    \item \textbf{3:} 
    \item \textbf{4:} 
    \item \textbf{5:} 
\end{itemize}

\chapter{Estado del Arte}

Se revisan las herramientas existentes para la evaluación de la huella de carbono del software, con un análisis de sus puntos fuertes y débiles. También se presentará una tabla resumen con las principales herramientas y sus características, lo que permitirá identificar las oportunidades para una nueva herramienta más realista y adaptada a los sistemas software actuales.

\chapter{Propuesta de implementación de EcoCode}

\section{Propuesta de valor}

\section{Diseño de la interfaz de usuario}

\chapter{Diseño Técnico}

\section{Historia de desarrollo}

\section{Diseño de pruebas}

\chapter{Documentación}

\section{Publicacion}

\chapter{Conclusiones y trabajos futuros}

% Bibliografía
\chapter*{Bibliografía}
\addcontentsline{toc}{chapter}{Bibliografía}
\begin{thebibliography}{99}
    \bibitem{ref1} Autor, A. (Año). Título del libro/artículo. Editorial/Revista.
    \bibitem{ref2} Autor, B. (Año). Título del artículo. Revista XYZ, Vol. 1, pp. 1--10.
\end{thebibliography}

\end{document}
